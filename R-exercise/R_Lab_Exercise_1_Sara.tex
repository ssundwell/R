% Options for packages loaded elsewhere
\PassOptionsToPackage{unicode}{hyperref}
\PassOptionsToPackage{hyphens}{url}
%
\documentclass[
]{article}
\usepackage{amsmath,amssymb}
\usepackage{iftex}
\ifPDFTeX
  \usepackage[T1]{fontenc}
  \usepackage[utf8]{inputenc}
  \usepackage{textcomp} % provide euro and other symbols
\else % if luatex or xetex
  \usepackage{unicode-math} % this also loads fontspec
  \defaultfontfeatures{Scale=MatchLowercase}
  \defaultfontfeatures[\rmfamily]{Ligatures=TeX,Scale=1}
\fi
\usepackage{lmodern}
\ifPDFTeX\else
  % xetex/luatex font selection
\fi
% Use upquote if available, for straight quotes in verbatim environments
\IfFileExists{upquote.sty}{\usepackage{upquote}}{}
\IfFileExists{microtype.sty}{% use microtype if available
  \usepackage[]{microtype}
  \UseMicrotypeSet[protrusion]{basicmath} % disable protrusion for tt fonts
}{}
\makeatletter
\@ifundefined{KOMAClassName}{% if non-KOMA class
  \IfFileExists{parskip.sty}{%
    \usepackage{parskip}
  }{% else
    \setlength{\parindent}{0pt}
    \setlength{\parskip}{6pt plus 2pt minus 1pt}}
}{% if KOMA class
  \KOMAoptions{parskip=half}}
\makeatother
\usepackage{xcolor}
\usepackage[margin=1in]{geometry}
\usepackage{color}
\usepackage{fancyvrb}
\newcommand{\VerbBar}{|}
\newcommand{\VERB}{\Verb[commandchars=\\\{\}]}
\DefineVerbatimEnvironment{Highlighting}{Verbatim}{commandchars=\\\{\}}
% Add ',fontsize=\small' for more characters per line
\usepackage{framed}
\definecolor{shadecolor}{RGB}{248,248,248}
\newenvironment{Shaded}{\begin{snugshade}}{\end{snugshade}}
\newcommand{\AlertTok}[1]{\textcolor[rgb]{0.94,0.16,0.16}{#1}}
\newcommand{\AnnotationTok}[1]{\textcolor[rgb]{0.56,0.35,0.01}{\textbf{\textit{#1}}}}
\newcommand{\AttributeTok}[1]{\textcolor[rgb]{0.13,0.29,0.53}{#1}}
\newcommand{\BaseNTok}[1]{\textcolor[rgb]{0.00,0.00,0.81}{#1}}
\newcommand{\BuiltInTok}[1]{#1}
\newcommand{\CharTok}[1]{\textcolor[rgb]{0.31,0.60,0.02}{#1}}
\newcommand{\CommentTok}[1]{\textcolor[rgb]{0.56,0.35,0.01}{\textit{#1}}}
\newcommand{\CommentVarTok}[1]{\textcolor[rgb]{0.56,0.35,0.01}{\textbf{\textit{#1}}}}
\newcommand{\ConstantTok}[1]{\textcolor[rgb]{0.56,0.35,0.01}{#1}}
\newcommand{\ControlFlowTok}[1]{\textcolor[rgb]{0.13,0.29,0.53}{\textbf{#1}}}
\newcommand{\DataTypeTok}[1]{\textcolor[rgb]{0.13,0.29,0.53}{#1}}
\newcommand{\DecValTok}[1]{\textcolor[rgb]{0.00,0.00,0.81}{#1}}
\newcommand{\DocumentationTok}[1]{\textcolor[rgb]{0.56,0.35,0.01}{\textbf{\textit{#1}}}}
\newcommand{\ErrorTok}[1]{\textcolor[rgb]{0.64,0.00,0.00}{\textbf{#1}}}
\newcommand{\ExtensionTok}[1]{#1}
\newcommand{\FloatTok}[1]{\textcolor[rgb]{0.00,0.00,0.81}{#1}}
\newcommand{\FunctionTok}[1]{\textcolor[rgb]{0.13,0.29,0.53}{\textbf{#1}}}
\newcommand{\ImportTok}[1]{#1}
\newcommand{\InformationTok}[1]{\textcolor[rgb]{0.56,0.35,0.01}{\textbf{\textit{#1}}}}
\newcommand{\KeywordTok}[1]{\textcolor[rgb]{0.13,0.29,0.53}{\textbf{#1}}}
\newcommand{\NormalTok}[1]{#1}
\newcommand{\OperatorTok}[1]{\textcolor[rgb]{0.81,0.36,0.00}{\textbf{#1}}}
\newcommand{\OtherTok}[1]{\textcolor[rgb]{0.56,0.35,0.01}{#1}}
\newcommand{\PreprocessorTok}[1]{\textcolor[rgb]{0.56,0.35,0.01}{\textit{#1}}}
\newcommand{\RegionMarkerTok}[1]{#1}
\newcommand{\SpecialCharTok}[1]{\textcolor[rgb]{0.81,0.36,0.00}{\textbf{#1}}}
\newcommand{\SpecialStringTok}[1]{\textcolor[rgb]{0.31,0.60,0.02}{#1}}
\newcommand{\StringTok}[1]{\textcolor[rgb]{0.31,0.60,0.02}{#1}}
\newcommand{\VariableTok}[1]{\textcolor[rgb]{0.00,0.00,0.00}{#1}}
\newcommand{\VerbatimStringTok}[1]{\textcolor[rgb]{0.31,0.60,0.02}{#1}}
\newcommand{\WarningTok}[1]{\textcolor[rgb]{0.56,0.35,0.01}{\textbf{\textit{#1}}}}
\usepackage{graphicx}
\makeatletter
\def\maxwidth{\ifdim\Gin@nat@width>\linewidth\linewidth\else\Gin@nat@width\fi}
\def\maxheight{\ifdim\Gin@nat@height>\textheight\textheight\else\Gin@nat@height\fi}
\makeatother
% Scale images if necessary, so that they will not overflow the page
% margins by default, and it is still possible to overwrite the defaults
% using explicit options in \includegraphics[width, height, ...]{}
\setkeys{Gin}{width=\maxwidth,height=\maxheight,keepaspectratio}
% Set default figure placement to htbp
\makeatletter
\def\fps@figure{htbp}
\makeatother
\setlength{\emergencystretch}{3em} % prevent overfull lines
\providecommand{\tightlist}{%
  \setlength{\itemsep}{0pt}\setlength{\parskip}{0pt}}
\setcounter{secnumdepth}{-\maxdimen} % remove section numbering
\ifLuaTeX
  \usepackage{selnolig}  % disable illegal ligatures
\fi
\usepackage{bookmark}
\IfFileExists{xurl.sty}{\usepackage{xurl}}{} % add URL line breaks if available
\urlstyle{same}
\hypersetup{
  pdftitle={DS311 - Basic R Lab Exercise},
  pdfauthor={Sara Noora Sundwell},
  hidelinks,
  pdfcreator={LaTeX via pandoc}}

\title{DS311 - Basic R Lab Exercise}
\usepackage{etoolbox}
\makeatletter
\providecommand{\subtitle}[1]{% add subtitle to \maketitle
  \apptocmd{\@title}{\par {\large #1 \par}}{}{}
}
\makeatother
\subtitle{R Lab Exercise}
\author{Sara Noora Sundwell}
\date{10/31/2024}

\begin{document}
\maketitle

\section{Basic R Exercise}\label{basic-r-exercise}

\subsection{Section 1 - Data Type}\label{section-1---data-type}

\textbf{Key Functions} - typeof() - as.numeric() - as.charater()

\subsubsection{Numeric}\label{numeric}

\begin{Shaded}
\begin{Highlighting}[]
\CommentTok{\# Numeric {-} Double precision by default}

\NormalTok{n1 }\OtherTok{\textless{}{-}} \DecValTok{15}  
\NormalTok{n1}
\end{Highlighting}
\end{Shaded}

\begin{verbatim}
## [1] 15
\end{verbatim}

\begin{Shaded}
\begin{Highlighting}[]
\FunctionTok{typeof}\NormalTok{(n1)}
\end{Highlighting}
\end{Shaded}

\begin{verbatim}
## [1] "double"
\end{verbatim}

\begin{Shaded}
\begin{Highlighting}[]
\NormalTok{n2 }\OtherTok{\textless{}{-}} \FloatTok{1.5}
\NormalTok{n2}
\end{Highlighting}
\end{Shaded}

\begin{verbatim}
## [1] 1.5
\end{verbatim}

\begin{Shaded}
\begin{Highlighting}[]
\FunctionTok{typeof}\NormalTok{(n2)}
\end{Highlighting}
\end{Shaded}

\begin{verbatim}
## [1] "double"
\end{verbatim}

\subsubsection{Character}\label{character}

\begin{Shaded}
\begin{Highlighting}[]
\CommentTok{\# Character}

\NormalTok{c1 }\OtherTok{\textless{}{-}} \StringTok{"c"}
\NormalTok{c1}
\end{Highlighting}
\end{Shaded}

\begin{verbatim}
## [1] "c"
\end{verbatim}

\begin{Shaded}
\begin{Highlighting}[]
\FunctionTok{typeof}\NormalTok{(c1)}
\end{Highlighting}
\end{Shaded}

\begin{verbatim}
## [1] "character"
\end{verbatim}

\begin{Shaded}
\begin{Highlighting}[]
\NormalTok{c2 }\OtherTok{\textless{}{-}} \StringTok{"a string of text"}
\NormalTok{c2}
\end{Highlighting}
\end{Shaded}

\begin{verbatim}
## [1] "a string of text"
\end{verbatim}

\begin{Shaded}
\begin{Highlighting}[]
\FunctionTok{typeof}\NormalTok{(c2)}
\end{Highlighting}
\end{Shaded}

\begin{verbatim}
## [1] "character"
\end{verbatim}

\subsubsection{Logical}\label{logical}

\begin{Shaded}
\begin{Highlighting}[]
\CommentTok{\# Logical}

\NormalTok{l1 }\OtherTok{\textless{}{-}} \ConstantTok{TRUE}
\NormalTok{l1}
\end{Highlighting}
\end{Shaded}

\begin{verbatim}
## [1] TRUE
\end{verbatim}

\begin{Shaded}
\begin{Highlighting}[]
\FunctionTok{typeof}\NormalTok{(l1)}
\end{Highlighting}
\end{Shaded}

\begin{verbatim}
## [1] "logical"
\end{verbatim}

\begin{Shaded}
\begin{Highlighting}[]
\NormalTok{l2 }\OtherTok{\textless{}{-}}\NormalTok{ F}
\NormalTok{l2}
\end{Highlighting}
\end{Shaded}

\begin{verbatim}
## [1] FALSE
\end{verbatim}

\begin{Shaded}
\begin{Highlighting}[]
\FunctionTok{typeof}\NormalTok{(l2)}
\end{Highlighting}
\end{Shaded}

\begin{verbatim}
## [1] "logical"
\end{verbatim}

\subsubsection{Transforming Numerics and
Characters}\label{transforming-numerics-and-characters}

\begin{Shaded}
\begin{Highlighting}[]
\CommentTok{\# Transforming numeric into characters}
\NormalTok{num }\OtherTok{\textless{}{-}} \DecValTok{10}
\NormalTok{numToChar }\OtherTok{\textless{}{-}} \FunctionTok{as.character}\NormalTok{(num)}
\FunctionTok{paste}\NormalTok{(}\StringTok{"num Type: "}\NormalTok{, }\FunctionTok{typeof}\NormalTok{(num), }\StringTok{" | numToChar: "}\NormalTok{, }\FunctionTok{typeof}\NormalTok{(numToChar))}
\end{Highlighting}
\end{Shaded}

\begin{verbatim}
## [1] "num Type:  double  | numToChar:  character"
\end{verbatim}

\begin{Shaded}
\begin{Highlighting}[]
\CommentTok{\# Transforming characters into numeric}
\NormalTok{char }\OtherTok{\textless{}{-}} \StringTok{"10"}
\NormalTok{charToNum }\OtherTok{\textless{}{-}} \FunctionTok{as.numeric}\NormalTok{(char)}
\FunctionTok{paste}\NormalTok{(}\StringTok{"char Type: "}\NormalTok{, }\FunctionTok{typeof}\NormalTok{(num), }\StringTok{" | charToNum: "}\NormalTok{, }\FunctionTok{typeof}\NormalTok{(numToChar))}
\end{Highlighting}
\end{Shaded}

\begin{verbatim}
## [1] "char Type:  double  | charToNum:  character"
\end{verbatim}

\subsubsection{Challenge:}\label{challenge}

Complete the following tasks:

\begin{Shaded}
\begin{Highlighting}[]
\CommentTok{\# Check the data type of the following variables}
\NormalTok{a }\OtherTok{\textless{}{-}} \FunctionTok{as.integer}\NormalTok{(}\DecValTok{500}\NormalTok{)}
\NormalTok{b }\OtherTok{\textless{}{-}} \FunctionTok{as.double}\NormalTok{(}\DecValTok{500}\NormalTok{)}
\NormalTok{c }\OtherTok{\textless{}{-}} \FunctionTok{as.character}\NormalTok{(}\DecValTok{500}\NormalTok{)}

\CommentTok{\# Enter your code here!}

\FunctionTok{typeof}\NormalTok{(a)}
\end{Highlighting}
\end{Shaded}

\begin{verbatim}
## [1] "integer"
\end{verbatim}

\begin{Shaded}
\begin{Highlighting}[]
\FunctionTok{typeof}\NormalTok{(b)}
\end{Highlighting}
\end{Shaded}

\begin{verbatim}
## [1] "double"
\end{verbatim}

\begin{Shaded}
\begin{Highlighting}[]
\FunctionTok{typeof}\NormalTok{(c)}
\end{Highlighting}
\end{Shaded}

\begin{verbatim}
## [1] "character"
\end{verbatim}

\begin{Shaded}
\begin{Highlighting}[]
\CommentTok{\# Check the data type of the new variable \textquotesingle{}d\textquotesingle{}}
\NormalTok{d }\OtherTok{\textless{}{-}}\NormalTok{ a }\SpecialCharTok{/}\NormalTok{ b}

\CommentTok{\# Enter your code here!}

\FunctionTok{typeof}\NormalTok{(d)}
\end{Highlighting}
\end{Shaded}

\begin{verbatim}
## [1] "double"
\end{verbatim}

\begin{center}\rule{0.5\linewidth}{0.5pt}\end{center}

\subsection{Section 2 - Data
Structure}\label{section-2---data-structure}

\begin{itemize}
\tightlist
\item
  is.vector()
\item
  is.matrix
\item
  cbind()
\item
  as.data.frame()
\end{itemize}

\subsubsection{Vector}\label{vector}

\begin{Shaded}
\begin{Highlighting}[]
\CommentTok{\# Vector}

\NormalTok{v1 }\OtherTok{\textless{}{-}} \FunctionTok{c}\NormalTok{(}\DecValTok{1}\NormalTok{, }\DecValTok{2}\NormalTok{, }\DecValTok{3}\NormalTok{, }\DecValTok{4}\NormalTok{, }\DecValTok{5}\NormalTok{)}
\NormalTok{v1}
\end{Highlighting}
\end{Shaded}

\begin{verbatim}
## [1] 1 2 3 4 5
\end{verbatim}

\begin{Shaded}
\begin{Highlighting}[]
\FunctionTok{is.vector}\NormalTok{(v1)}
\end{Highlighting}
\end{Shaded}

\begin{verbatim}
## [1] TRUE
\end{verbatim}

\begin{Shaded}
\begin{Highlighting}[]
\NormalTok{v2 }\OtherTok{\textless{}{-}} \FunctionTok{c}\NormalTok{(}\StringTok{"a"}\NormalTok{, }\StringTok{"b"}\NormalTok{, }\StringTok{"c"}\NormalTok{)}
\NormalTok{v2}
\end{Highlighting}
\end{Shaded}

\begin{verbatim}
## [1] "a" "b" "c"
\end{verbatim}

\begin{Shaded}
\begin{Highlighting}[]
\FunctionTok{is.vector}\NormalTok{(v2)}
\end{Highlighting}
\end{Shaded}

\begin{verbatim}
## [1] TRUE
\end{verbatim}

\begin{Shaded}
\begin{Highlighting}[]
\NormalTok{v3 }\OtherTok{\textless{}{-}} \FunctionTok{c}\NormalTok{(}\ConstantTok{TRUE}\NormalTok{, }\ConstantTok{TRUE}\NormalTok{, }\ConstantTok{FALSE}\NormalTok{, }\ConstantTok{FALSE}\NormalTok{, }\ConstantTok{TRUE}\NormalTok{)}
\NormalTok{v3}
\end{Highlighting}
\end{Shaded}

\begin{verbatim}
## [1]  TRUE  TRUE FALSE FALSE  TRUE
\end{verbatim}

\begin{Shaded}
\begin{Highlighting}[]
\FunctionTok{is.vector}\NormalTok{(v3)}
\end{Highlighting}
\end{Shaded}

\begin{verbatim}
## [1] TRUE
\end{verbatim}

\subsubsection{Matrix}\label{matrix}

\begin{Shaded}
\begin{Highlighting}[]
\CommentTok{\# Matrix}

\NormalTok{m1 }\OtherTok{\textless{}{-}} \FunctionTok{matrix}\NormalTok{(}\FunctionTok{c}\NormalTok{(T, T, F, F, T, F), }\AttributeTok{nrow =} \DecValTok{2}\NormalTok{)}
\NormalTok{m1}
\end{Highlighting}
\end{Shaded}

\begin{verbatim}
##      [,1]  [,2]  [,3]
## [1,] TRUE FALSE  TRUE
## [2,] TRUE FALSE FALSE
\end{verbatim}

\begin{Shaded}
\begin{Highlighting}[]
\FunctionTok{is.matrix}\NormalTok{(m1)}
\end{Highlighting}
\end{Shaded}

\begin{verbatim}
## [1] TRUE
\end{verbatim}

\begin{Shaded}
\begin{Highlighting}[]
\NormalTok{m2 }\OtherTok{\textless{}{-}} \FunctionTok{matrix}\NormalTok{(}\FunctionTok{c}\NormalTok{(}\StringTok{"a"}\NormalTok{, }\StringTok{"b"}\NormalTok{, }
               \StringTok{"c"}\NormalTok{, }\StringTok{"d"}\NormalTok{), }
               \AttributeTok{nrow =} \DecValTok{2}\NormalTok{,}
               \AttributeTok{byrow =}\NormalTok{ T)}
\NormalTok{m2}
\end{Highlighting}
\end{Shaded}

\begin{verbatim}
##      [,1] [,2]
## [1,] "a"  "b" 
## [2,] "c"  "d"
\end{verbatim}

\begin{Shaded}
\begin{Highlighting}[]
\FunctionTok{is.matrix}\NormalTok{(m2)}
\end{Highlighting}
\end{Shaded}

\begin{verbatim}
## [1] TRUE
\end{verbatim}

\subsubsection{Challenge:}\label{challenge-1}

\begin{enumerate}
\def\labelenumi{\arabic{enumi}.}
\tightlist
\item
  Create a vector of the 26 alphabet lower case letters in sequence.
\item
  Create a 2 by 13 matrix for the 26 English upper case letter in
  sequence.
\end{enumerate}

Hint: Check out the ``letters'' and ``LETTERS'' key words in R.

\begin{Shaded}
\begin{Highlighting}[]
\CommentTok{\# Enter your code here.}

\NormalTok{lowercase\_letters }\OtherTok{\textless{}{-}}\NormalTok{ letters}
\FunctionTok{print}\NormalTok{(lowercase\_letters)}
\end{Highlighting}
\end{Shaded}

\begin{verbatim}
##  [1] "a" "b" "c" "d" "e" "f" "g" "h" "i" "j" "k" "l" "m" "n" "o" "p" "q" "r" "s"
## [20] "t" "u" "v" "w" "x" "y" "z"
\end{verbatim}

\begin{Shaded}
\begin{Highlighting}[]
\NormalTok{uppercase\_matrix }\OtherTok{\textless{}{-}} \FunctionTok{matrix}\NormalTok{(LETTERS,}
                           \AttributeTok{nrow =} \DecValTok{2}\NormalTok{,}
                           \AttributeTok{ncol =} \DecValTok{13}\NormalTok{,}
                           \AttributeTok{byrow =}\NormalTok{ T)}
\FunctionTok{print}\NormalTok{(uppercase\_matrix)}
\end{Highlighting}
\end{Shaded}

\begin{verbatim}
##      [,1] [,2] [,3] [,4] [,5] [,6] [,7] [,8] [,9] [,10] [,11] [,12] [,13]
## [1,] "A"  "B"  "C"  "D"  "E"  "F"  "G"  "H"  "I"  "J"   "K"   "L"   "M"  
## [2,] "N"  "O"  "P"  "Q"  "R"  "S"  "T"  "U"  "V"  "W"   "X"   "Y"   "Z"
\end{verbatim}

\subsubsection{DataFrame}\label{dataframe}

\begin{Shaded}
\begin{Highlighting}[]
\CommentTok{\# Data Frame}

\CommentTok{\# Can combine vectors of the same length}
\NormalTok{vNumeric   }\OtherTok{\textless{}{-}} \FunctionTok{c}\NormalTok{(}\DecValTok{1}\NormalTok{, }\DecValTok{2}\NormalTok{, }\DecValTok{3}\NormalTok{)}
\NormalTok{vCharacter }\OtherTok{\textless{}{-}} \FunctionTok{c}\NormalTok{(}\StringTok{"a"}\NormalTok{, }\StringTok{"b"}\NormalTok{, }\StringTok{"c"}\NormalTok{)}
\NormalTok{vLogical   }\OtherTok{\textless{}{-}} \FunctionTok{c}\NormalTok{(T, F, T)}

\NormalTok{df1 }\OtherTok{\textless{}{-}} \FunctionTok{cbind}\NormalTok{(vNumeric, vCharacter, vLogical)}
\NormalTok{df1  }\CommentTok{\# Coerces all values to most basic data type}
\end{Highlighting}
\end{Shaded}

\begin{verbatim}
##      vNumeric vCharacter vLogical
## [1,] "1"      "a"        "TRUE"  
## [2,] "2"      "b"        "FALSE" 
## [3,] "3"      "c"        "TRUE"
\end{verbatim}

\begin{Shaded}
\begin{Highlighting}[]
\NormalTok{df2 }\OtherTok{\textless{}{-}} \FunctionTok{as.data.frame}\NormalTok{(}\FunctionTok{cbind}\NormalTok{(vNumeric, vCharacter, vLogical))}
\NormalTok{df2  }\CommentTok{\# Makes a data frame with three different data types}
\end{Highlighting}
\end{Shaded}

\begin{verbatim}
##   vNumeric vCharacter vLogical
## 1        1          a     TRUE
## 2        2          b    FALSE
## 3        3          c     TRUE
\end{verbatim}

\begin{center}\rule{0.5\linewidth}{0.5pt}\end{center}

\subsection{Section 3 - Setup Working Directory and Installing
Packages}\label{section-3---setup-working-directory-and-installing-packages}

\textbf{Key Functions:} - getwd() - setwd() - install.packages() -
library()

\subsubsection{Setting up your working
directory}\label{setting-up-your-working-directory}

\begin{Shaded}
\begin{Highlighting}[]
\CommentTok{\# Check your current working directory}
\CommentTok{\# wd1 \textless{}{-} getwd()}
\CommentTok{\# paste("Current Working Directory: ", wd1)}

\CommentTok{\# Setting the working directory for a project}
\CommentTok{\# setwd("c://.../project")}
\CommentTok{\# wd2 \textless{}{-} getwd()}
\CommentTok{\# paste("Current Working Directory: ", wd2)}
\end{Highlighting}
\end{Shaded}

\subsubsection{Installing and Loading
Packages}\label{installing-and-loading-packages}

\begin{center}\rule{0.5\linewidth}{0.5pt}\end{center}

\subsection{Section 4 - Problem
Solving}\label{section-4---problem-solving}

Write the code that accomplish the following tasks:

Part a: Assign 4 to variable x

Part b: Assign 12 to variable y

Part c: Print both x and y to check their values

Part d: Divide y by x and assign it to variable z

part e: Print a statement to report your answer in Part d.

Once you finished and knit the RMarkdown file into html file, you should
be able to see the message ``Congratulation!! You completed the first
exercise in this section!!'' in the html document.

\begin{Shaded}
\begin{Highlighting}[]
\CommentTok{\# Write your code here!}
\CommentTok{\# Part a}

\NormalTok{x }\OtherTok{\textless{}{-}} \DecValTok{4}

\CommentTok{\# Part b}

\NormalTok{y}\OtherTok{\textless{}{-}}\DecValTok{12}

\CommentTok{\# Part c}

\FunctionTok{print}\NormalTok{(x)}
\end{Highlighting}
\end{Shaded}

\begin{verbatim}
## [1] 4
\end{verbatim}

\begin{Shaded}
\begin{Highlighting}[]
\FunctionTok{print}\NormalTok{(y)}
\end{Highlighting}
\end{Shaded}

\begin{verbatim}
## [1] 12
\end{verbatim}

\begin{Shaded}
\begin{Highlighting}[]
\CommentTok{\#or to print them both at the same time}

\FunctionTok{cat}\NormalTok{(x,y)}
\end{Highlighting}
\end{Shaded}

\begin{verbatim}
## 4 12
\end{verbatim}

\begin{Shaded}
\begin{Highlighting}[]
\CommentTok{\# Part d}

\NormalTok{z}\OtherTok{\textless{}{-}}\NormalTok{y}\SpecialCharTok{/}\NormalTok{x}

\CommentTok{\# Part e}
\FunctionTok{print}\NormalTok{(}\FunctionTok{paste}\NormalTok{(}\StringTok{"y divided by x is equal to "}\NormalTok{, z))}
\end{Highlighting}
\end{Shaded}

\begin{verbatim}
## [1] "y divided by x is equal to  3"
\end{verbatim}

\begin{Shaded}
\begin{Highlighting}[]
\CommentTok{\# Do not need to change the following code!}
\ControlFlowTok{if}\NormalTok{ (}\FunctionTok{exists}\NormalTok{(}\StringTok{"x"}\NormalTok{) }\SpecialCharTok{==} \ConstantTok{TRUE} \SpecialCharTok{|} \FunctionTok{exists}\NormalTok{(}\StringTok{"y"}\NormalTok{) }\SpecialCharTok{==} \ConstantTok{TRUE} \SpecialCharTok{|} \FunctionTok{exists}\NormalTok{(}\StringTok{"z"}\NormalTok{) }\SpecialCharTok{==} \ConstantTok{TRUE}\NormalTok{)\{}
  \ControlFlowTok{if}\NormalTok{ (x }\SpecialCharTok{==} \DecValTok{4} \SpecialCharTok{\&}\NormalTok{ y }\SpecialCharTok{==} \DecValTok{12} \SpecialCharTok{\&}\NormalTok{ z }\SpecialCharTok{==} \DecValTok{3}\NormalTok{) \{}
  \FunctionTok{print}\NormalTok{(}\StringTok{"Congratulation!!  You completed the first activity in this class!!"}\NormalTok{)}
\NormalTok{  \} }\ControlFlowTok{else}\NormalTok{ \{}
    \FunctionTok{print}\NormalTok{(}\StringTok{"Sorry, you got it wrong!"}\NormalTok{)}
\NormalTok{  \}}
\NormalTok{\} }\ControlFlowTok{else}\NormalTok{ \{}
  \FunctionTok{print}\NormalTok{(}\StringTok{"You did not complete the last problem!"}\NormalTok{)}
\NormalTok{\}}
\end{Highlighting}
\end{Shaded}

\begin{verbatim}
## [1] "Congratulation!!  You completed the first activity in this class!!"
\end{verbatim}

\end{document}
