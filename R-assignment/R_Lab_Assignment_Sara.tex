% Options for packages loaded elsewhere
\PassOptionsToPackage{unicode}{hyperref}
\PassOptionsToPackage{hyphens}{url}
%
\documentclass[
]{article}
\usepackage{amsmath,amssymb}
\usepackage{iftex}
\ifPDFTeX
  \usepackage[T1]{fontenc}
  \usepackage[utf8]{inputenc}
  \usepackage{textcomp} % provide euro and other symbols
\else % if luatex or xetex
  \usepackage{unicode-math} % this also loads fontspec
  \defaultfontfeatures{Scale=MatchLowercase}
  \defaultfontfeatures[\rmfamily]{Ligatures=TeX,Scale=1}
\fi
\usepackage{lmodern}
\ifPDFTeX\else
  % xetex/luatex font selection
\fi
% Use upquote if available, for straight quotes in verbatim environments
\IfFileExists{upquote.sty}{\usepackage{upquote}}{}
\IfFileExists{microtype.sty}{% use microtype if available
  \usepackage[]{microtype}
  \UseMicrotypeSet[protrusion]{basicmath} % disable protrusion for tt fonts
}{}
\makeatletter
\@ifundefined{KOMAClassName}{% if non-KOMA class
  \IfFileExists{parskip.sty}{%
    \usepackage{parskip}
  }{% else
    \setlength{\parindent}{0pt}
    \setlength{\parskip}{6pt plus 2pt minus 1pt}}
}{% if KOMA class
  \KOMAoptions{parskip=half}}
\makeatother
\usepackage{xcolor}
\usepackage[margin=1in]{geometry}
\usepackage{color}
\usepackage{fancyvrb}
\newcommand{\VerbBar}{|}
\newcommand{\VERB}{\Verb[commandchars=\\\{\}]}
\DefineVerbatimEnvironment{Highlighting}{Verbatim}{commandchars=\\\{\}}
% Add ',fontsize=\small' for more characters per line
\usepackage{framed}
\definecolor{shadecolor}{RGB}{248,248,248}
\newenvironment{Shaded}{\begin{snugshade}}{\end{snugshade}}
\newcommand{\AlertTok}[1]{\textcolor[rgb]{0.94,0.16,0.16}{#1}}
\newcommand{\AnnotationTok}[1]{\textcolor[rgb]{0.56,0.35,0.01}{\textbf{\textit{#1}}}}
\newcommand{\AttributeTok}[1]{\textcolor[rgb]{0.13,0.29,0.53}{#1}}
\newcommand{\BaseNTok}[1]{\textcolor[rgb]{0.00,0.00,0.81}{#1}}
\newcommand{\BuiltInTok}[1]{#1}
\newcommand{\CharTok}[1]{\textcolor[rgb]{0.31,0.60,0.02}{#1}}
\newcommand{\CommentTok}[1]{\textcolor[rgb]{0.56,0.35,0.01}{\textit{#1}}}
\newcommand{\CommentVarTok}[1]{\textcolor[rgb]{0.56,0.35,0.01}{\textbf{\textit{#1}}}}
\newcommand{\ConstantTok}[1]{\textcolor[rgb]{0.56,0.35,0.01}{#1}}
\newcommand{\ControlFlowTok}[1]{\textcolor[rgb]{0.13,0.29,0.53}{\textbf{#1}}}
\newcommand{\DataTypeTok}[1]{\textcolor[rgb]{0.13,0.29,0.53}{#1}}
\newcommand{\DecValTok}[1]{\textcolor[rgb]{0.00,0.00,0.81}{#1}}
\newcommand{\DocumentationTok}[1]{\textcolor[rgb]{0.56,0.35,0.01}{\textbf{\textit{#1}}}}
\newcommand{\ErrorTok}[1]{\textcolor[rgb]{0.64,0.00,0.00}{\textbf{#1}}}
\newcommand{\ExtensionTok}[1]{#1}
\newcommand{\FloatTok}[1]{\textcolor[rgb]{0.00,0.00,0.81}{#1}}
\newcommand{\FunctionTok}[1]{\textcolor[rgb]{0.13,0.29,0.53}{\textbf{#1}}}
\newcommand{\ImportTok}[1]{#1}
\newcommand{\InformationTok}[1]{\textcolor[rgb]{0.56,0.35,0.01}{\textbf{\textit{#1}}}}
\newcommand{\KeywordTok}[1]{\textcolor[rgb]{0.13,0.29,0.53}{\textbf{#1}}}
\newcommand{\NormalTok}[1]{#1}
\newcommand{\OperatorTok}[1]{\textcolor[rgb]{0.81,0.36,0.00}{\textbf{#1}}}
\newcommand{\OtherTok}[1]{\textcolor[rgb]{0.56,0.35,0.01}{#1}}
\newcommand{\PreprocessorTok}[1]{\textcolor[rgb]{0.56,0.35,0.01}{\textit{#1}}}
\newcommand{\RegionMarkerTok}[1]{#1}
\newcommand{\SpecialCharTok}[1]{\textcolor[rgb]{0.81,0.36,0.00}{\textbf{#1}}}
\newcommand{\SpecialStringTok}[1]{\textcolor[rgb]{0.31,0.60,0.02}{#1}}
\newcommand{\StringTok}[1]{\textcolor[rgb]{0.31,0.60,0.02}{#1}}
\newcommand{\VariableTok}[1]{\textcolor[rgb]{0.00,0.00,0.00}{#1}}
\newcommand{\VerbatimStringTok}[1]{\textcolor[rgb]{0.31,0.60,0.02}{#1}}
\newcommand{\WarningTok}[1]{\textcolor[rgb]{0.56,0.35,0.01}{\textbf{\textit{#1}}}}
\usepackage{graphicx}
\makeatletter
\def\maxwidth{\ifdim\Gin@nat@width>\linewidth\linewidth\else\Gin@nat@width\fi}
\def\maxheight{\ifdim\Gin@nat@height>\textheight\textheight\else\Gin@nat@height\fi}
\makeatother
% Scale images if necessary, so that they will not overflow the page
% margins by default, and it is still possible to overwrite the defaults
% using explicit options in \includegraphics[width, height, ...]{}
\setkeys{Gin}{width=\maxwidth,height=\maxheight,keepaspectratio}
% Set default figure placement to htbp
\makeatletter
\def\fps@figure{htbp}
\makeatother
\setlength{\emergencystretch}{3em} % prevent overfull lines
\providecommand{\tightlist}{%
  \setlength{\itemsep}{0pt}\setlength{\parskip}{0pt}}
\setcounter{secnumdepth}{-\maxdimen} % remove section numbering
\ifLuaTeX
  \usepackage{selnolig}  % disable illegal ligatures
\fi
\usepackage{bookmark}
\IfFileExists{xurl.sty}{\usepackage{xurl}}{} % add URL line breaks if available
\urlstyle{same}
\hypersetup{
  pdftitle={DS311 - R Lab Assignment},
  pdfauthor={Sara Noora Sundwell},
  hidelinks,
  pdfcreator={LaTeX via pandoc}}

\title{DS311 - R Lab Assignment}
\author{Sara Noora Sundwell}
\date{2024-11-04}

\begin{document}
\maketitle

\subsection{R Assignment 1}\label{r-assignment-1}

\begin{itemize}
\tightlist
\item
  In this assignment, we are going to apply some of the build in data
  set in R for descriptive statistics analysis.
\item
  To earn full grade in this assignment, students need to complete the
  coding tasks for each question to get the result.
\item
  After finished all the questions, knit the document into HTML format
  for submission.
\end{itemize}

\subsubsection{Question 1}\label{question-1}

Using the \textbf{mtcars} data set in R, please answer the following
questions.

\begin{Shaded}
\begin{Highlighting}[]
\CommentTok{\# Loading the data}
\FunctionTok{data}\NormalTok{(mtcars)}

\CommentTok{\# Head of the data set}
\FunctionTok{head}\NormalTok{(mtcars)}
\end{Highlighting}
\end{Shaded}

\begin{verbatim}
##                    mpg cyl disp  hp drat    wt  qsec vs am gear carb
## Mazda RX4         21.0   6  160 110 3.90 2.620 16.46  0  1    4    4
## Mazda RX4 Wag     21.0   6  160 110 3.90 2.875 17.02  0  1    4    4
## Datsun 710        22.8   4  108  93 3.85 2.320 18.61  1  1    4    1
## Hornet 4 Drive    21.4   6  258 110 3.08 3.215 19.44  1  0    3    1
## Hornet Sportabout 18.7   8  360 175 3.15 3.440 17.02  0  0    3    2
## Valiant           18.1   6  225 105 2.76 3.460 20.22  1  0    3    1
\end{verbatim}

\begin{enumerate}
\def\labelenumi{\alph{enumi}.}
\tightlist
\item
  Report the number of variables and observations in the data set.
\end{enumerate}

\begin{Shaded}
\begin{Highlighting}[]
\CommentTok{\# Enter your code here!}

\NormalTok{Variables }\OtherTok{\textless{}{-}} \FunctionTok{ncol}\NormalTok{(mtcars)}
\NormalTok{Observations }\OtherTok{\textless{}{-}} \FunctionTok{nrow}\NormalTok{(mtcars)}

\CommentTok{\# Answer:}
\FunctionTok{print}\NormalTok{(}\FunctionTok{paste}\NormalTok{(}\StringTok{"There are total of"}\NormalTok{, Variables, }\StringTok{"variables and"}\NormalTok{, Observations, }\StringTok{"observations in this data set."}\NormalTok{))}
\end{Highlighting}
\end{Shaded}

\begin{verbatim}
## [1] "There are total of 11 variables and 32 observations in this data set."
\end{verbatim}

\begin{enumerate}
\def\labelenumi{\alph{enumi}.}
\setcounter{enumi}{1}
\tightlist
\item
  Print the summary statistics of the data set and report how many
  discrete and continuous variables are in the data set.
\end{enumerate}

\begin{Shaded}
\begin{Highlighting}[]
\CommentTok{\# Enter your code here!}

\FunctionTok{summary}\NormalTok{(mtcars)}
\end{Highlighting}
\end{Shaded}

\begin{verbatim}
##       mpg             cyl             disp             hp       
##  Min.   :10.40   Min.   :4.000   Min.   : 71.1   Min.   : 52.0  
##  1st Qu.:15.43   1st Qu.:4.000   1st Qu.:120.8   1st Qu.: 96.5  
##  Median :19.20   Median :6.000   Median :196.3   Median :123.0  
##  Mean   :20.09   Mean   :6.188   Mean   :230.7   Mean   :146.7  
##  3rd Qu.:22.80   3rd Qu.:8.000   3rd Qu.:326.0   3rd Qu.:180.0  
##  Max.   :33.90   Max.   :8.000   Max.   :472.0   Max.   :335.0  
##       drat             wt             qsec             vs        
##  Min.   :2.760   Min.   :1.513   Min.   :14.50   Min.   :0.0000  
##  1st Qu.:3.080   1st Qu.:2.581   1st Qu.:16.89   1st Qu.:0.0000  
##  Median :3.695   Median :3.325   Median :17.71   Median :0.0000  
##  Mean   :3.597   Mean   :3.217   Mean   :17.85   Mean   :0.4375  
##  3rd Qu.:3.920   3rd Qu.:3.610   3rd Qu.:18.90   3rd Qu.:1.0000  
##  Max.   :4.930   Max.   :5.424   Max.   :22.90   Max.   :1.0000  
##        am              gear            carb      
##  Min.   :0.0000   Min.   :3.000   Min.   :1.000  
##  1st Qu.:0.0000   1st Qu.:3.000   1st Qu.:2.000  
##  Median :0.0000   Median :4.000   Median :2.000  
##  Mean   :0.4062   Mean   :3.688   Mean   :2.812  
##  3rd Qu.:1.0000   3rd Qu.:4.000   3rd Qu.:4.000  
##  Max.   :1.0000   Max.   :5.000   Max.   :8.000
\end{verbatim}

\begin{Shaded}
\begin{Highlighting}[]
\FunctionTok{str}\NormalTok{(mtcars)}
\end{Highlighting}
\end{Shaded}

\begin{verbatim}
## 'data.frame':    32 obs. of  11 variables:
##  $ mpg : num  21 21 22.8 21.4 18.7 18.1 14.3 24.4 22.8 19.2 ...
##  $ cyl : num  6 6 4 6 8 6 8 4 4 6 ...
##  $ disp: num  160 160 108 258 360 ...
##  $ hp  : num  110 110 93 110 175 105 245 62 95 123 ...
##  $ drat: num  3.9 3.9 3.85 3.08 3.15 2.76 3.21 3.69 3.92 3.92 ...
##  $ wt  : num  2.62 2.88 2.32 3.21 3.44 ...
##  $ qsec: num  16.5 17 18.6 19.4 17 ...
##  $ vs  : num  0 0 1 1 0 1 0 1 1 1 ...
##  $ am  : num  1 1 1 0 0 0 0 0 0 0 ...
##  $ gear: num  4 4 4 3 3 3 3 4 4 4 ...
##  $ carb: num  4 4 1 1 2 1 4 2 2 4 ...
\end{verbatim}

\begin{Shaded}
\begin{Highlighting}[]
\CommentTok{\# Answer:}
\FunctionTok{print}\NormalTok{(}\FunctionTok{paste}\NormalTok{(}\StringTok{"There are"}\NormalTok{, }\FunctionTok{ncol}\NormalTok{(mtcars) }\SpecialCharTok{{-}} \FunctionTok{sum}\NormalTok{(}\FunctionTok{sapply}\NormalTok{(mtcars, is.numeric)), }\StringTok{"discrete variables and"}\NormalTok{, }\FunctionTok{sum}\NormalTok{(}\FunctionTok{sapply}\NormalTok{(mtcars, is.numeric)), }\StringTok{"continuous variables in this data set."}\NormalTok{))}
\end{Highlighting}
\end{Shaded}

\begin{verbatim}
## [1] "There are 0 discrete variables and 11 continuous variables in this data set."
\end{verbatim}

\begin{enumerate}
\def\labelenumi{\alph{enumi}.}
\setcounter{enumi}{2}
\tightlist
\item
  Calculate the mean, variance, and standard deviation for the variable
  \textbf{mpg} and assign them into variable names m, v, and s. Report
  the results in the print statement.
\end{enumerate}

\begin{Shaded}
\begin{Highlighting}[]
\CommentTok{\# Enter your code here!}

\NormalTok{m }\OtherTok{\textless{}{-}} \FunctionTok{mean}\NormalTok{(mtcars}\SpecialCharTok{$}\NormalTok{mpg, }\AttributeTok{na.rm =} \ConstantTok{TRUE}\NormalTok{)}
\NormalTok{v }\OtherTok{\textless{}{-}} \FunctionTok{var}\NormalTok{(mtcars}\SpecialCharTok{$}\NormalTok{mpg, }\AttributeTok{na.rm =} \ConstantTok{TRUE}\NormalTok{)}
\NormalTok{s }\OtherTok{\textless{}{-}} \FunctionTok{sd}\NormalTok{(mtcars}\SpecialCharTok{$}\NormalTok{mpg, }\AttributeTok{na.rm =} \ConstantTok{TRUE}\NormalTok{)}


\CommentTok{\# Answer}
\FunctionTok{print}\NormalTok{(}\FunctionTok{paste}\NormalTok{(}\StringTok{"The average of Mile Per Gallon from this data set is "}\NormalTok{, m , }\StringTok{" with variance "}\NormalTok{, v , }\StringTok{" and standard deviation"}\NormalTok{, s , }\StringTok{"."}\NormalTok{))}
\end{Highlighting}
\end{Shaded}

\begin{verbatim}
## [1] "The average of Mile Per Gallon from this data set is  20.090625  with variance  36.3241028225806  and standard deviation 6.0269480520891 ."
\end{verbatim}

\begin{enumerate}
\def\labelenumi{\alph{enumi}.}
\setcounter{enumi}{3}
\tightlist
\item
  Create two tables to summarize 1) average mpg for each cylinder class
  and 2) the standard deviation of mpg for each gear class.
\end{enumerate}

\begin{Shaded}
\begin{Highlighting}[]
\CommentTok{\# Enter your code here!}

\CommentTok{\#Table 1}

\NormalTok{average\_mpg\_cylinder }\OtherTok{\textless{}{-}} \FunctionTok{aggregate}\NormalTok{(mpg }\SpecialCharTok{\textasciitilde{}}\NormalTok{ cyl, }\AttributeTok{data =}\NormalTok{ mtcars, }\AttributeTok{FUN =}\NormalTok{ mean)}


\CommentTok{\#Table 2}

\NormalTok{std\_dev\_mpg\_gear }\OtherTok{\textless{}{-}} \FunctionTok{aggregate}\NormalTok{(mpg }\SpecialCharTok{\textasciitilde{}}\NormalTok{ gear, }\AttributeTok{data =}\NormalTok{ mtcars, }\AttributeTok{FUN =}\NormalTok{ sd)}

\CommentTok{\#Print tables}
\FunctionTok{print}\NormalTok{(}\StringTok{"Average MPG for each Cylinder Class:"}\NormalTok{)}
\end{Highlighting}
\end{Shaded}

\begin{verbatim}
## [1] "Average MPG for each Cylinder Class:"
\end{verbatim}

\begin{Shaded}
\begin{Highlighting}[]
\FunctionTok{print}\NormalTok{(average\_mpg\_cylinder)}
\end{Highlighting}
\end{Shaded}

\begin{verbatim}
##   cyl      mpg
## 1   4 26.66364
## 2   6 19.74286
## 3   8 15.10000
\end{verbatim}

\begin{Shaded}
\begin{Highlighting}[]
\FunctionTok{print}\NormalTok{(}\StringTok{"Standard Deviation of MPG for each Gear Class:"}\NormalTok{)}
\end{Highlighting}
\end{Shaded}

\begin{verbatim}
## [1] "Standard Deviation of MPG for each Gear Class:"
\end{verbatim}

\begin{Shaded}
\begin{Highlighting}[]
\FunctionTok{print}\NormalTok{(std\_dev\_mpg\_gear)}
\end{Highlighting}
\end{Shaded}

\begin{verbatim}
##   gear      mpg
## 1    3 3.371618
## 2    4 5.276764
## 3    5 6.658979
\end{verbatim}

\begin{enumerate}
\def\labelenumi{\alph{enumi}.}
\setcounter{enumi}{4}
\tightlist
\item
  Create a crosstab that shows the number of observations belong to each
  cylinder and gear class combinations. The table should show how many
  observations given the car has 4 cylinders with 3 gears, 4 cylinders
  with 4 gears, etc. Report which combination is recorded in this data
  set and how many observations for this type of car.
\end{enumerate}

\begin{Shaded}
\begin{Highlighting}[]
\CommentTok{\# Enter your code here!}

\NormalTok{table }\OtherTok{\textless{}{-}} \FunctionTok{table}\NormalTok{(mtcars}\SpecialCharTok{$}\NormalTok{cyl, mtcars}\SpecialCharTok{$}\NormalTok{gear)}

\CommentTok{\# Print the crosstab}
\FunctionTok{print}\NormalTok{(table)}
\end{Highlighting}
\end{Shaded}

\begin{verbatim}
##    
##      3  4  5
##   4  1  8  2
##   6  2  4  1
##   8 12  0  2
\end{verbatim}

\begin{Shaded}
\begin{Highlighting}[]
\CommentTok{\# Find the most common car type}
\NormalTok{max\_count }\OtherTok{\textless{}{-}} \FunctionTok{max}\NormalTok{(table)}
\NormalTok{type }\OtherTok{\textless{}{-}} \FunctionTok{which}\NormalTok{(table }\SpecialCharTok{==}\NormalTok{ max\_count, }\AttributeTok{arr.ind =} \ConstantTok{TRUE}\NormalTok{)}

\FunctionTok{print}\NormalTok{(}\FunctionTok{paste}\NormalTok{(}\StringTok{"The most common car type in this data set is car with"}\NormalTok{, type[}\DecValTok{1}\NormalTok{], }\StringTok{"cylinders and"}\NormalTok{, type[}\DecValTok{2}\NormalTok{], }\StringTok{"gears. There are total of"}\NormalTok{, max\_count, }\StringTok{"cars belong to this specification in the data set."}\NormalTok{))}
\end{Highlighting}
\end{Shaded}

\begin{verbatim}
## [1] "The most common car type in this data set is car with 3 cylinders and 1 gears. There are total of 12 cars belong to this specification in the data set."
\end{verbatim}

\begin{center}\rule{0.5\linewidth}{0.5pt}\end{center}

\subsubsection{Question 2}\label{question-2}

Use different visualization tools to summarize the data sets in this
question.

\begin{enumerate}
\def\labelenumi{\alph{enumi}.}
\tightlist
\item
  Using the \textbf{PlantGrowth} data set, visualize and compare the
  weight of the plant in the three separated group. Give labels to the
  title, x-axis, and y-axis on the graph. Write a paragraph to summarize
  your findings.
\end{enumerate}

\begin{Shaded}
\begin{Highlighting}[]
\CommentTok{\# Load the data set}
\FunctionTok{data}\NormalTok{(}\StringTok{"PlantGrowth"}\NormalTok{)}

\CommentTok{\# Head of the data set}
\FunctionTok{head}\NormalTok{(PlantGrowth)}
\end{Highlighting}
\end{Shaded}

\begin{verbatim}
##   weight group
## 1   4.17  ctrl
## 2   5.58  ctrl
## 3   5.18  ctrl
## 4   6.11  ctrl
## 5   4.50  ctrl
## 6   4.61  ctrl
\end{verbatim}

\begin{Shaded}
\begin{Highlighting}[]
\CommentTok{\# Enter your code here!}

\FunctionTok{boxplot}\NormalTok{(weight }\SpecialCharTok{\textasciitilde{}}\NormalTok{ group, }\AttributeTok{data =}\NormalTok{ PlantGrowth,}
        \AttributeTok{main =} \StringTok{"Plant Growth by Group"}\NormalTok{,}
        \AttributeTok{xlab =} \StringTok{"Group"}\NormalTok{, }\AttributeTok{ylab =} \StringTok{"Weight"}\NormalTok{,}
        \AttributeTok{names =} \FunctionTok{c}\NormalTok{(}\StringTok{"Control"}\NormalTok{, }\StringTok{"Treatment 1"}\NormalTok{, }\StringTok{"Treatment 2"}\NormalTok{))}
\end{Highlighting}
\end{Shaded}

\includegraphics{R_Lab_Assignment_Sara_files/figure-latex/unnamed-chunk-7-1.pdf}

\begin{Shaded}
\begin{Highlighting}[]
\CommentTok{\# The boxplot shows that:}
\CommentTok{\#         Group 1 (Control) exhibits the highest median plant weight, followed by Group 2 (Treatment 1)           and Group 3 (Treatment 2)}

\CommentTok{\#         The variation in weight appears to be greatest in Group 1, while Groups 2 and 3 show less              variability.}
\CommentTok{\#}
\end{Highlighting}
\end{Shaded}

\begin{enumerate}
\def\labelenumi{\alph{enumi}.}
\setcounter{enumi}{1}
\tightlist
\item
  Using the \textbf{mtcars} data set, plot the histogram for the column
  \textbf{mpg} with 10 breaks. Give labels to the title, x-axis, and
  y-axis on the graph. Report the most observed mpg class from the data
  set.
\end{enumerate}

\begin{Shaded}
\begin{Highlighting}[]
\CommentTok{\# Print histogram}
\FunctionTok{hist}\NormalTok{(mtcars}\SpecialCharTok{$}\NormalTok{mpg, }\AttributeTok{breaks =} \DecValTok{10}\NormalTok{,}
     \AttributeTok{main =} \StringTok{"Histogram of Miles Per Gallon"}\NormalTok{,}
     \AttributeTok{xlab =} \StringTok{"Miles Per Gallon"}\NormalTok{, }\AttributeTok{ylab =} \StringTok{"Frequency"}\NormalTok{)}
\end{Highlighting}
\end{Shaded}

\includegraphics{R_Lab_Assignment_Sara_files/figure-latex/unnamed-chunk-8-1.pdf}

\begin{Shaded}
\begin{Highlighting}[]
\CommentTok{\# Observations}
\NormalTok{most\_observed }\OtherTok{\textless{}{-}} \FunctionTok{as.numeric}\NormalTok{(}\FunctionTok{names}\NormalTok{(}\FunctionTok{sort}\NormalTok{(}\FunctionTok{table}\NormalTok{(mtcars}\SpecialCharTok{$}\NormalTok{mpg), }\AttributeTok{decreasing =} \ConstantTok{TRUE}\NormalTok{)[}\DecValTok{1}\NormalTok{]))}

\CommentTok{\# Answer}
\FunctionTok{print}\NormalTok{(}\FunctionTok{paste}\NormalTok{(}\StringTok{"Most of the cars in this data set are in the class of"}\NormalTok{, most\_observed, }\StringTok{"mile per gallon."}\NormalTok{))}
\end{Highlighting}
\end{Shaded}

\begin{verbatim}
## [1] "Most of the cars in this data set are in the class of 10.4 mile per gallon."
\end{verbatim}

\begin{enumerate}
\def\labelenumi{\alph{enumi}.}
\setcounter{enumi}{2}
\tightlist
\item
  Using the \textbf{USArrests} data set, create a pairs plot to display
  the correlations between the variables in the data set. Plot the
  scatter plot with \textbf{Murder} and \textbf{Assault}. Give labels to
  the title, x-axis, and y-axis on the graph. Write a paragraph to
  summarize your results from both plots.
\end{enumerate}

\begin{Shaded}
\begin{Highlighting}[]
\CommentTok{\# Load the data set}
\FunctionTok{data}\NormalTok{(}\StringTok{"USArrests"}\NormalTok{)}

\CommentTok{\# Head of the data set}
\FunctionTok{head}\NormalTok{(USArrests)}
\end{Highlighting}
\end{Shaded}

\begin{verbatim}
##            Murder Assault UrbanPop Rape
## Alabama      13.2     236       58 21.2
## Alaska       10.0     263       48 44.5
## Arizona       8.1     294       80 31.0
## Arkansas      8.8     190       50 19.5
## California    9.0     276       91 40.6
## Colorado      7.9     204       78 38.7
\end{verbatim}

\begin{Shaded}
\begin{Highlighting}[]
\CommentTok{\# Enter your code here!}
\FunctionTok{pairs}\NormalTok{(USArrests)}
\end{Highlighting}
\end{Shaded}

\includegraphics{R_Lab_Assignment_Sara_files/figure-latex/unnamed-chunk-9-1.pdf}

\begin{Shaded}
\begin{Highlighting}[]
\FunctionTok{plot}\NormalTok{(USArrests}\SpecialCharTok{$}\NormalTok{Murder, USArrests}\SpecialCharTok{$}\NormalTok{Assault,}
     \AttributeTok{main =} \StringTok{"Murder vs. Assault"}\NormalTok{,}
     \AttributeTok{xlab =} \StringTok{"Murder Rate"}\NormalTok{, }\AttributeTok{ylab =} \StringTok{"Assault Rate"}\NormalTok{)}
\end{Highlighting}
\end{Shaded}

\includegraphics{R_Lab_Assignment_Sara_files/figure-latex/unnamed-chunk-9-2.pdf}

\begin{Shaded}
\begin{Highlighting}[]
\CommentTok{\# My findings}
\CommentTok{\#   The plots show a strong link between murder and assault rates in US states.}
\CommentTok{\#   States with high murder rates often have high assault rates too.}
\end{Highlighting}
\end{Shaded}

\begin{center}\rule{0.5\linewidth}{0.5pt}\end{center}

\subsubsection{Question 3}\label{question-3}

Download the housing data set from www.jaredlander.com and find out what
explains the housing prices in New York City.

Note: Check your working directory to make sure that you can download
the data into the data folder.

\begin{enumerate}
\def\labelenumi{\alph{enumi}.}
\tightlist
\item
  Create your own descriptive statistics and aggregation tables to
  summarize the data set and find any meaningful results between
  different variables in the data set.
\end{enumerate}

\begin{Shaded}
\begin{Highlighting}[]
\CommentTok{\# Head of the cleaned data set}
\FunctionTok{head}\NormalTok{(housingData)}
\end{Highlighting}
\end{Shaded}

\begin{verbatim}
##   Neighborhood Market.Value.per.SqFt      Boro Year.Built
## 1    FINANCIAL                200.00 Manhattan       1920
## 2    FINANCIAL                242.76 Manhattan       1985
## 4    FINANCIAL                271.23 Manhattan       1930
## 5      TRIBECA                247.48 Manhattan       1985
## 6      TRIBECA                191.37 Manhattan       1986
## 7      TRIBECA                211.53 Manhattan       1985
\end{verbatim}

\begin{Shaded}
\begin{Highlighting}[]
\CommentTok{\# Enter your code here!}
\FunctionTok{summary}\NormalTok{(housingData)}
\end{Highlighting}
\end{Shaded}

\begin{verbatim}
##  Neighborhood       Market.Value.per.SqFt     Boro             Year.Built  
##  Length:2530        Min.   : 10.66        Length:2530        Min.   :1825  
##  Class :character   1st Qu.: 75.10        Class :character   1st Qu.:1926  
##  Mode  :character   Median :114.89        Mode  :character   Median :1986  
##                     Mean   :133.17                           Mean   :1967  
##                     3rd Qu.:189.91                           3rd Qu.:2005  
##                     Max.   :399.38                           Max.   :2010
\end{verbatim}

\begin{Shaded}
\begin{Highlighting}[]
\CommentTok{\# Average price per square foot by borough}
\NormalTok{avg\_price\_by\_boro }\OtherTok{\textless{}{-}} \FunctionTok{aggregate}\NormalTok{(Market.Value.per.SqFt }\SpecialCharTok{\textasciitilde{}}\NormalTok{ Boro, }\AttributeTok{data =}\NormalTok{ housingData, }\AttributeTok{FUN =}\NormalTok{ mean)}
\FunctionTok{print}\NormalTok{(avg\_price\_by\_boro)}
\end{Highlighting}
\end{Shaded}

\begin{verbatim}
##            Boro Market.Value.per.SqFt
## 1         Bronx              47.93232
## 2      Brooklyn              80.13439
## 3     Manhattan             180.59265
## 4        Queens              77.38137
## 5 Staten Island              41.26958
\end{verbatim}

\begin{Shaded}
\begin{Highlighting}[]
\CommentTok{\# Correlation between Year Built and Market Value per SqFt}
\FunctionTok{cor}\NormalTok{(housingData}\SpecialCharTok{$}\NormalTok{Year.Built, housingData}\SpecialCharTok{$}\NormalTok{Market.Value.per.SqFt)}
\end{Highlighting}
\end{Shaded}

\begin{verbatim}
## [1] -0.09559073
\end{verbatim}

\begin{Shaded}
\begin{Highlighting}[]
\CommentTok{\# Boxplot of Market Value per SqFt by Borough}
\FunctionTok{boxplot}\NormalTok{(Market.Value.per.SqFt }\SpecialCharTok{\textasciitilde{}}\NormalTok{ Boro, }\AttributeTok{data =}\NormalTok{ housingData, }\AttributeTok{main =} \StringTok{"Market Value per SqFt by Borough"}\NormalTok{)}
\end{Highlighting}
\end{Shaded}

\includegraphics{R_Lab_Assignment_Sara_files/figure-latex/unnamed-chunk-11-1.pdf}

\begin{enumerate}
\def\labelenumi{\alph{enumi}.}
\setcounter{enumi}{1}
\tightlist
\item
  Create multiple plots to demonstrates the correlations between
  different variables. Remember to label all axes and give title to each
  graph.
\end{enumerate}

\begin{Shaded}
\begin{Highlighting}[]
\CommentTok{\# Enter your code here!}
\FunctionTok{plot}\NormalTok{(housingData}\SpecialCharTok{$}\NormalTok{Year.Built, housingData}\SpecialCharTok{$}\NormalTok{Market.Value.per.SqFt,}
     \AttributeTok{xlab =} \StringTok{"Year Built"}\NormalTok{, }\AttributeTok{ylab =} \StringTok{"Market Value per SqFt"}\NormalTok{,}
     \AttributeTok{main =} \StringTok{"Market Value vs. Year Built"}\NormalTok{)}
\end{Highlighting}
\end{Shaded}

\includegraphics{R_Lab_Assignment_Sara_files/figure-latex/unnamed-chunk-12-1.pdf}

\begin{Shaded}
\begin{Highlighting}[]
\CommentTok{\# Boxplot of market value by borough}
\FunctionTok{boxplot}\NormalTok{(Market.Value.per.SqFt }\SpecialCharTok{\textasciitilde{}}\NormalTok{ Boro, }\AttributeTok{data =}\NormalTok{ housingData,}
        \AttributeTok{main =} \StringTok{"Market Value by Borough"}\NormalTok{)}
\end{Highlighting}
\end{Shaded}

\includegraphics{R_Lab_Assignment_Sara_files/figure-latex/unnamed-chunk-12-2.pdf}

\begin{enumerate}
\def\labelenumi{\alph{enumi}.}
\setcounter{enumi}{2}
\tightlist
\item
  Write a summary about your findings from this exercise.
\end{enumerate}

=\textgreater{} Enter your answer here!

So, what did we learn about NYC housing? - New buildings cost more:
Newer buildings are generally more expensive than older ones. -
Manhattan is the most expensive: If you want to live in Manhattan, be
prepared to pay a lot of money. - Other boroughs are cheaper: The other
boroughs in NYC are generally more affordable than Manhattan.

Basically, if you want a fancy, new apartment in Manhattan, get ready to
break the bank

\end{document}
